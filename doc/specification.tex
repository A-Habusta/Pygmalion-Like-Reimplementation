\documentclass{article}
\usepackage[utf8]{inputenc}
\usepackage[english]{babel}
\usepackage{amsfonts}
\usepackage{fancyhdr}
\usepackage{hyperref}
\usepackage{titling}

\renewcommand\maketitlehooka{\null\mbox{}\vfill}
\renewcommand\maketitlehookd{\vfill\null}

\newcommand{\currentversion}{0.0.1}

%extremely hacky solution right now

\author{%
    \vspace{3.5mm}\currentversion\\
    Adrián Habušta
    }
\title{%
    \LARGE Software Project Specification \\
    \Large for \\
    \Huge Reimplementation of Pygmalion \\\vspace{10mm}
    \normalsize Pygmalion was a proof of concept visual programming system
    put forward in the year 1975. It involved programming using so-called
    icons, and creating functions by remembering user actions, instead of
    writing them in code. This project focuses on creating a simple programming
    system with the same ideas as Pygmalion.
    }
\date{May 4, 2023}

\begin{document}
    \begin{titlepage}
        \maketitle
    \end{titlepage}

    \tableofcontents
    \newpage

    \section{Basic Information}
        \subsection{Description of Software Project}
            This software project is a programming system inspired by an old project called Pygmalion.
            Pygmalion never moved past the prototype stage, and this project is a proof of concept that is meant
            to show how using a system with ideas borrowed from Pygmalion would be like. These ideas include
            representing data and methods with icons, and creating functions by remembering user actions.

        \subsection{Used technologies}
            \begin{itemize}
                \item F\# - used programming language
                \item Fable - F\# to JS compiler
            \end{itemize}

        \subsection{References}
            \begin{itemize}
                \item Pygmalion specification
                    \subitem - \url{https://apps.dtic.mil/sti/pdfs/ADA016811.pdf}
            \end{itemize}


    \section{Detailed functionality}
        \subsection{Data representation}


    \section{Screens}
        \subsection{Main Screen}
        \subsection{Icon Designer}

    \section{Usage example - Factorial}
        This section will demonstrate how an icon which computes the factorial of a number can be created.\\

        We open the program and see the main screen\\

        We click the \textbf{Create New Icon} button below the custom icons menu.\\

        We name our icon 'factorial'. This creates a new tab to which we are moved. \\



    \section{Other requirements}


    \section{Timeline \& Milestones}
        \begin{tabular}{ | c | c | c | }
            \hline
            Date & Milestone & Presentation method\\
            \hline

            14.05.2023 & First version of documentation & Meeting with supervisor \\

            % Add more milestones after meeting

            \hline

        \end{tabular}


    \addcontentsline{toc}{section}{Appendix A: Terminology}
    \section*{Appendix A: Terminology}


\end{document}
