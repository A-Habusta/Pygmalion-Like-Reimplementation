\documentclass[r]{article}
\documentclass[titlepge]{report}
\usepackage[utf8]{inputenc}
\usepackage[english]{babel}
\usepackage{amsfonts}
\usepackage{fancyhdr}
\usepackage{titling}


\usepackage{titling}
\renewcommand\maketitlehooka{\null\mbox{}\vfill}
\renewcommand\maketitlehookd{\vfill\null}

\newcommand{\currentversion}{0.0.1}

%extremely hacky solution right now

\author{%
    \vspace{3.5mm}\currentversion\\
    Adrián Habušta
    }
\title{%
    \LARGE Software Project Specification \\
    \Large for \\
    \Huge Reimplementation of Pygmalion \\\vspace{10mm}
    \normalsize Pygmalion was a proof of concept visual programming system
    put forward in the year 1975. It involved programming using so-called
    icons, and creating functions by remembering user actions, instead of
    writing them in code. This project focuses on creating a simple programming
    system with the same ideas as Pygmalion.
    }
\date{May 4, 2023}

\begin{document}
    \begin{titlepage}
        \maketitle
    \end{titlepage}

    \tableofcontents
    \newpage

    \section{Basic Information}
        \subsection{Description of Software Project}

        \subsection{Used technologies}

        \subsection{References}


    \section{Short description of Software Project}
        \subsection{Reason for creation, major parts and goals}


    \section{Detailed functionality}
        \subsection{Data representation}


    \section{Screens}
        \subsection{Main Screen}
        \subsection{Icon Designer}

    \section{Usage example}
        \subsection{Factorial}


    \section{Other requirements}


    \section{Milestones}


    \addcontentsline{toc}{section}{Appendix A: Terminology}
    \section*{Appendix A: Terminology}


\end{document}
