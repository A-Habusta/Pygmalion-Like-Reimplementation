\documentclass{article}
\usepackage[utf8]{inputenc}
\usepackage[english]{babel}
\usepackage{amsfonts}
\usepackage{fancyhdr}
\usepackage{hyperref}
\usepackage{titling}

\renewcommand\maketitlehooka{\null\mbox{}\vfill}
\renewcommand\maketitlehookd{\vfill\null}

\newcommand{\currentversion}{0.1.0}

%extremely hacky solution right now

\author{%
    \vspace{3.5mm}\currentversion\\
    Adrián Habušta
    }
\title{%
    \LARGE Software Project Specification \\
    \Large for \\
    \Huge Reimplementation of Pygmalion \\\vspace{10mm}
    \normalsize Pygmalion was a proof of concept visual programming system
    put forward in the year 1975. It involved programming using so-called
    icons, and creating functions by remembering user actions, instead of
    writing them in code. This project focuses on creating a simple programming
    system with the same key ideas as Pygmalion.
    }
\date{May 14, 2023}

\begin{document}
    \begin{titlepage}
        \maketitle
    \end{titlepage}

    \tableofcontents
    \newpage

    \section{Basic Information}
        \subsection{Description of Software Project}
            This software project is a programming system inspired by an old project called Pygmalion.
            Pygmalion never moved past the prototype stage, and this project is a proof of concept that is meant
            to show how using a system with ideas borrowed from Pygmalion would be like. These ideas include
            representing data, methods and even objects with icons. Another key idea is creating methods by remembering
            user actions. This is what this project will focus on the most.

        \subsection{Used technologies}
            \begin{itemize}
                \item
                    F\# - used programming language
                \item
                    Fable - F\# to JS compiler

            \end{itemize}

        \subsection{References}
            \begin{itemize}

                \item
                    Pygmalion specification
                    \subitem
                        - \url{https://apps.dtic.mil/sti/pdfs/ADA016811.pdf}

            \end{itemize}


    \section{Detailed functionality}
        \subsection{Icons representation}
            Internally, icons will consist of only 



    \section{Screens}
        \subsection{Single Tab}
        This is the only screen the software will feature. There will, however, exist multiple instances of this screen
        called 'tabs' that can be switched between.
        % Insert picture here


    \section{Usage example - Factorial}
        This section will demonstrate how an icon which computes the factorial of a number can be created.\\

        We open the program and see the main screen\\

        We click the \textbf{Create New Icon} button below the custom icons menu.\\

        We name our icon 'factorial'. This creates a new empty icon, with said name, which is added to the custom icons
        menu. This icon can than be dragged onto the field. \\

        With the icon on the field, we can fill its inputs with values, and then left-click on the icon to execute its
        code. \\

        Since the icon is new, we immediately hit a so called \textbf{trap}. This pauses execution and opens a new tab
        containing the \textbf{context} of the icon (which is mostly empty for new icons). This is where we program
        icons. \\

        The context of an icon reprents the internals of an icon, which are used to compute the value of an icon. We
        can add icons to this context by dragging them onto the screen on this new tab. We can add all the necessary
        icons for computing the factorial now.\\

        \Large TODO \normalsize




    \section{Other requirements}
        Because of the fact that this project is a proof of concept, there are not many requirements placed
        on aspects of the project such as performance. The only requirement is that the software works on modern
        browsers, and is responsive.

    \section{Project restrictions}
        \begin{itemize}

            \item
                The only supported type that can be used within the icons are integers. The project may be extended
                to support arbitrary types later on, but for now, this is a non-goal.

            \item
                The icons can only represent pure functions. There is no internal state that can be referenced. This
                means that icons representing data can only exist as functions returning a constant, and that creating
                icons which represent objects is impossible as of now.

        \end{itemize}


    \section{Timeline \& Milestones}
        \begin{tabular}{ | c | c | c | }
            \hline
            Date & Milestone & Presentation method\\
            \hline

            14.05.2023 & First version of documentation & Meeting with supervisor \\

            % Add more milestones after meeting

            \hline

        \end{tabular}


    \addcontentsline{toc}{section}{Appendix A: Terminology}

    \section*{Appendix A: Terminology}


\end{document}
